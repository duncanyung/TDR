\section{Experiment}\label{sec:exp}


\subsection{Experiment Design}

\subsubsection{Fast Cycle}
In this experiment, we want to evaluate the effectiveness of our outliner detection model in Section~\ref{sec:outliner}. 
We want to show that:
\begin{enumerate}
 \item most people are healthy people and they would have similar $F_c$, and 
 \item most people can approximately determine their $F_c$. 
\end{enumerate}


Since we are unable to obtain real $F_c$ data, in this experiment, we will use similar data to evaluate the 
effectiveness of our proposed model. 
% We ask workers in Amazon Mechanical Turk to understand what are Tongue and sweaty 
We perform our experiment in Amazon Mechanical Turk. 
Given 5 pictures of 5 different faces, we ask 1000 workers to determine the size of nose, eyes, and mouth in every picture. 
For all 5 pictures, persons in the pictures have similar size of nose, eyes, and mouth. 
Worker can offer a score from 1 to 5, where 1 is the smallest size and 5 is the biggest size. 
Then, we put all data points together and build a model using method in Section~\ref{sec:outliner}. 
We want to show that it is possible to find a dense region in all data points, which implies 
most people would have similar sense while a small number of people are different from majority and people are able to 
offer approximate score for things that they understand. 


\subsubsection{Slow Cycle}
{\bf To Prof. Liang, could you please fill up this sub-section?}
Since no real data, in order to make a good social network learning model, we design a similar experiment named gray-scale learning.
(1)Definiton: two kinds of person, one is master(answers always is right), another are students(keep learning for improving judgement).
(2)Basic knowledge: gray-scale has 256 value, from 0 to 255. Before the expereiemnt, all students are given a pitcure with 5 
 five colors which show five different gray values.
(3)learning process: In every learning cycle, the master and students are given a new picture which contains five random colors.  Students need to 
fill out gray values for every colors based on their basic knowledge and knoelwedge learnt in previous cycles. 
The master also has to fill out gray values for all five random colors. The answer offered by the master are given to all students at 
the end of this cycle so that students can learn from the master's answer, which is always treated as the right answer. 
After got the right answer from the master, they will enter next slow cycle. 
(4)We will record every judgement result for every students and the master, and the records will be used for evaluating the effectiveness of 
the social network model. 




